\documentclass[pdftex,12pt,a4paper]{report}

\usepackage[portuguese,english]{babel}
\usepackage[T1]{fontenc} 
\usepackage[table,xcdraw]{xcolor}
\usepackage[utf8]{inputenc}
\usepackage[pdftex]{graphicx}
\usepackage{minitoc}
\usepackage{hyperref}
\usepackage{indentfirst}
\usepackage[compact]{titlesec}
\usepackage{fancyhdr}
\usepackage{caption}
\usepackage{pgfplots}
\usepackage{pgfplotstable}
\usepackage{fixltx2e}
\usepackage{mathtools}
\usepackage{fancyhdr}
\usepackage{listings}
\usepackage{color}
\usepackage{sverb}
\usepackage[section]{placeins}
\usepackage{adjustbox}
\titleformat*{\subsubsection}{\itshape}


%Highlight
\newcommand{\shellcmd}[1]{\indent\indent\texttt{\footnotesize\# #1}\\}

\pagestyle{fancy}
\renewcommand*\thesection{\thechapter\arabic{section}}
\newcommand{\HRule}{\rule{\linewidth}{0.5mm}}
\begin{document}

\begin{titlepage}

\begin{center}

\includegraphics[width=0.15\textwidth]{./logo}\\[0.5cm]    

\textsc{\large Universidade de Aveiro \\[1cm]\large departamento de electrónica, telecomunicações e informática}\\[1cm]

\textsc{\large{47064}\large - Desempenho e Dimensionamento de Redes\\[1cm]}

\HRule \\[0.5cm]
{ \huge \bfseries Network Traffic Engineering}\\[0.4cm]
\HRule \\[1cm]

\textsc{\small{8240 - MESTRADO INTEGRADO EM ENGENHARIA DE COMPUTADORES E TELEMÁTICA}}\\[1cm]

\begin{minipage}{0.4\textwidth}

\begin{flushleft} \large
\href{mailto:rafael.ferreira@ua.pt}{António Rafael da \\ Costa Ferreira }
 \small{\\NMec: 67405}
\end{flushleft}
\end{minipage}
\begin{minipage}{0.4\textwidth}

\begin{flushright} \large
\href{mailto:rodrigocunha@ua.pt}{Rodrigo Lopes \\ da Cunha}
\small{\\NMec: 67800}
\end{flushright}
\end{minipage}\\[1cm]

{\large Docentes: Paulo Salvador,}\\
{\large \hspace{2.5cm} Susana Sargento}\\[0.5cm]


\vfill

{\large Maio de 2016 \\ 2015-2016}

\end{center}

\end{titlepage} %Titulo do Relatorio
\renewcommand{\headrulewidth}{0pt}

%Cabeçalhos de rodapé
\fancyhead{}
\fancyfoot{}
\lhead{Network Traffic Engineering}
\rhead{DDR - 2015/2016}
\lfoot{Rafael Ferreira nmec: 67405 \\ Rodrigo Cunha nmec: 67800}
\rfoot{\thepage}

%Renomear Comandos
\renewcommand*\contentsname{Conteúdos}
\renewcommand*\figurename{Figura}
\renewcommand*\tablename{Tabela}

%Conteúdos, dar paragrafo
\tableofcontents
%Headers
\renewcommand{\headrulewidth}{0.15pt}
\renewcommand{\thechapter}{}

\clearpage

\section{Virtual-Circuit Switched Network}

\subsection{Exercício 1 e 2}

Nestes primeiros exercícios era pedido que se obtivesse os valores de probabilidade de bloqueio e de carga média do link, tanto de forma simulada como teórica, para vários valores de $\lambda$ (média de pedidos de chamadas por minuto), 1/$\mu$ (média da duração de cada chamada, em minutos), onde cada circuito virtual (VC), requer uma largura de banda de 2 Mbits/sec e o link pode tomar vários valores de largura de banda.


Foram obtidos os seguintes resultados para cada valor de largura de banda do link:

Nos três casos, é possível observar-se o valor simulado (linha contínua) e o valor teórico (linha picotada), para os vários tempos médios de duração de uma chamada.

Através da figura 1, onde a largura de banda do link é de 16 Mbits/seg é possível ver que a probabilidade de bloqueio é muito maior, comparando com a figura 3, onde a largura de banda é o dobro, 32 Mbits/sec. Como existe mais largura de banda no link, para um mesmo número de pedidos de chamadas, estas bloqueiam em menos quantidade, pois é possível processar um maior número de pedidos. 

Em todos os gráficos, é possível observar-se que à medida que a taxa de chamadas vai aumentando, a probabilidade de bloqueio e a ocupação média do link também vão aumentar, pois são mais pedidos para processar.

Relativamente à duração das chamadas, quando esta aumenta, a probabilidade de bloqueio e a ocupação média do link também vão aumentar, pois as chamadas ocuparão por mais tempo o link, com a mesma taxa de pedidos.

\newpage
\subsection{Exercício 3}

Neste terceiro exercício, as chamadas passaram a ser de dois tipos, um standard e um special. As chamadas standard requerem uma largura de banda de 2 Mbits/sec e as special requerem uma largura de banda de 4 Mbits/sec. Neste caso, os nós possuem ainda uma quantidade de Mbit/s reservados para os pedidos especiais. Foram obtidos os resultados da simulação para vários valores de $\lambda$ standard e special, 1/$\mu$, largura de banda do link e largura de banda reservada.

Obtiveram-se os seguintes resultados, através da execução de 3 amostras:

Observando os gráficos acima, é possível observar-se que quanto maior for a capacidade do link, os pedidos standard, começam a bloquear menos vezes. É possível verificar que quanto maior o valor de largura de banda reservada, melhores serão os pedidos special, e pior os standard. 

Na figura 4, o link possuí uma largura de banda de 16 Mbits/sec, e a largura de banda reservada, sendo 16 Mbits/sec, faz com que os pedidos standard bloqueiem 100\% das vezes, pois toda a largura de banda existente está disponível apenas para os pedidos special.

Nas figuras 5 e 6, o valor da largura de banda do link vai aumentado. Com isto, a probabilidade de bloqueio dos pedidos standard vai diminuindo. A maior diminuição verifica-se quando a duração dos pedidos é a mais curta, linha azul, isto porque, tal como se verificou no primeiro exercício quanto mais baixa for a duração do pedido, menor será a probabilidade de bloqueio.

Com o aumento da largura de banda do link, os pedidos especiais também terão menor probabilidade de bloqueio, tal como se verificou no exercício 1.

Para este exercício foram criados dois geradores de pedidos, um para pedidos standard, e outro para pedidos special.

\newpage
\end{document}



