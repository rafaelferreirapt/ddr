\documentclass[pdftex,12pt,a4paper]{report}

\usepackage[portuguese,english]{babel}
\usepackage[T1]{fontenc} 
\usepackage[table,xcdraw]{xcolor}
\usepackage[utf8]{inputenc}
\usepackage[pdftex]{graphicx}
\usepackage{minitoc}
\usepackage{hyperref}
\usepackage{indentfirst}
\usepackage[compact]{titlesec}
\usepackage{fancyhdr}
\usepackage{caption}
\usepackage{pgfplots}
\usepackage{pgfplotstable}
\usepackage{fixltx2e}
\usepackage{mathtools}
\usepackage{fancyhdr}
\usepackage{listings}
\usepackage{color}
\usepackage{sverb}
\usepackage[section]{placeins}
\usepackage{adjustbox}
\titleformat*{\subsubsection}{\itshape}


%Highlight
\newcommand{\shellcmd}[1]{\indent\indent\texttt{\footnotesize\# #1}\\}

\pagestyle{fancy}
\renewcommand*\thesection{\thechapter\arabic{section}}
\newcommand{\HRule}{\rule{\linewidth}{0.5mm}}
\begin{document}

\begin{titlepage}

\begin{center}

\includegraphics[width=0.15\textwidth]{./logo}\\[0.5cm]    

\textsc{\large Universidade de Aveiro \\[1cm]\large departamento de electrónica, telecomunicações e informática}\\[1cm]

\textsc{\large{47064}\large - Desempenho e Dimensionamento de Redes\\[1cm]}

\HRule \\[0.5cm]
{ \huge \bfseries Network Traffic Engineering}\\[0.4cm]
\HRule \\[1cm]

\textsc{\small{8240 - MESTRADO INTEGRADO EM ENGENHARIA DE COMPUTADORES E TELEMÁTICA}}\\[1cm]

\begin{minipage}{0.4\textwidth}

\begin{flushleft} \large
\href{mailto:rafael.ferreira@ua.pt}{António Rafael da \\ Costa Ferreira }
 \small{\\NMec: 67405}
\end{flushleft}
\end{minipage}
\begin{minipage}{0.4\textwidth}

\begin{flushright} \large
\href{mailto:rodrigocunha@ua.pt}{Rodrigo Lopes \\ da Cunha}
\small{\\NMec: 67800}
\end{flushright}
\end{minipage}\\[1cm]

{\large Docentes: Paulo Salvador,}\\
{\large \hspace{2.5cm} Susana Sargento}\\[0.5cm]


\vfill

{\large Maio de 2016 \\ 2015-2016}

\end{center}

\end{titlepage} %Titulo do Relatorio
\renewcommand{\headrulewidth}{0pt}

%Cabeçalhos de rodapé
\fancyhead{}
\fancyfoot{}
\lhead{Network Traffic Engineering}
\rhead{DDR - 2015/2016}
\lfoot{Rafael Ferreira nmec: 67405 \\ Rodrigo Cunha nmec: 67800}
\rfoot{\thepage}

%Renomear Comandos
\renewcommand*\contentsname{Conteúdos}
\renewcommand*\figurename{Figura}
\renewcommand*\tablename{Tabela}

%Conteúdos, dar paragrafo
\tableofcontents
%Headers
\renewcommand{\headrulewidth}{0.15pt}
\renewcommand{\thechapter}{}



\clearpage

\section{Exercício 1}

Para iniciar este trabalho foi-nos dado vários ficheiros de base. O ficheiro NetGen.py é responsável por gerar uma rede, com vários nós que inclui o nome e localização geográfica, e as ligações inter-nó, matrizes de tráfego que define os fluxos de tráfego entre todas as cidades/ nós.

A localização geográfica (pos) é obtida dinamicamente a partir do nome do no (nome da cidade) usando a API do Google Maps, ficheiro getGeo.py. 

A matriz de tráfego (TM) é gerado aleatóriamente. É possível também guardar o resultado num ficheiro .dat e passando como argumento o parâmetro -f, net.dat.

\begin{figure}[!htb]
  \centering
  \begin{minipage}[b]{0.4\textwidth}
    \includegraphics[width=\textwidth]{imagensGuia/esquema_rede_pequena.png}
  \end{minipage}
  \hfill
  \begin{minipage}[b]{0.4\textwidth}
    \includegraphics[width=\textwidth]{imagensGuia/esquema_rede_grande.png}
  \end{minipage}
  \caption{Rede pequena de teste e rede grande}
\end{figure}


\section{Exercício 2, 3 e 4}

No exercício 2 é usado o caminho mais curto como escolha para o caminho entre pontos, sendo que isto é dado pela soma das conexões e usando o algoritmo Greedy.

No exercício 3, foi calculado o "average one-way delay" e a carga em todas as direções em todos os links.

Para isso, para calcular o "average one-way delay", foi usada a seguinte fórmula baseada na aproximação Kleinrock:

$\mu$ = 1e9 / 8000 , é igual ao link speed em pkts/sec (1Gbps)

\[W = 1e6\times\left[\frac{1}{\left(\mu-atraso\right)}\right]\]

Para calcular o atraso, teve de se criar um ciclo de forma a percorrer todos os links e criar uma lista com os atrasos.

\begin{lstlisting}[language=python]
for pair in allpairs:
    path = sol[pair]
    for i in range(0, len(path) - 1):
        ws_delay[pair] = 1e6 / (mu - \
        net[path[i]][path[i + 1]]['load'])
\end{lstlisting}

Após isso foi possível apresentar a seguinte tabela para a rede pequena:

\begin{table}[!htb]
        \centering
        \resizebox{\textwidth}{!}{\begin{tabular}{|l|l|l|l|l|}
\hline
\multicolumn{1}{|c|}{\textbf{Origem}} & \multicolumn{1}{|c|}{\textbf{Destino}} & \multicolumn{1}{|c|}{\textbf{Saltos}} & \multicolumn{1}{|c|}{\textbf{Carga (pkts/sec)}} & \multicolumn{1}{|c|}{\textbf{Atraso (micro/sec)}} \\ \hline
Lisboa & Viseu & Lisboa, Viseu & 31271 & 10.67 \\ \hline
Porto & Lisboa & Porto, Aveiro, Lisboa & Indisponível & 15.95 \\ \hline
Viseu & Porto & Viseu, Porto & 31401 & 10.68 \\ \hline
Lisboa & Aveiro & Lisboa, Aveiro & 62675 & 16.04 \\ \hline
Aveiro & Viseu & Aveiro, Viseu & 31199 & 10.66 \\ \hline
Viseu & Aveiro & Viseu, Aveiro & 31378 & 10.68 \\ \hline
Aveiro & Porto & Aveiro, Porto & 63050 & 16.14 \\ \hline
Porto & Viseu & Porto, Viseu & 31396 & 10.68 \\ \hline
Porto & Aveiro & Porto, Aveiro & 62129 & 15.91 \\ \hline
Lisboa & Porto & Lisboa, Aveiro, Porto & Indisponível & 16.14 \\ \hline
Viseu & Lisboa & Viseu, Lisboa & 31171 & 10.66 \\ \hline
Aveiro & Lisboa & Aveiro, Lisboa & 62304 & 15.95 \\ \hline
\end{tabular}}
\caption[]{Solução obtida, carga nos links e atraso}
\end{table}



Analisando a tabela obtida conseguimos perceber que o caminho Lisboa-Porto tem o máximo delay de 32.19 micro segundos, carga média de 16.01 micro segundos, a carga máxima está de Aveiro ao Porto com 63050 pacotes/ segundo e carga média de 43797.40 pacotes/ segundo.

\begin{table}[!htb]
        \centering
        \resizebox{\textwidth}{!}{\begin{tabular}{|l|l|l|}
\hline
\multicolumn{1}{|c|}{\textbf{Maximum one-way delay flow}} & \multicolumn{1}{|c|}{\textbf{Maximum one-way delay}} & \multicolumn{1}{|c|}{\textbf{Mean one-way delay}} \\ \hline
Aveiro-Porto & 16.1420500404 & 13.3475508848 \\ \hline
\end{tabular}}
\caption[]{Atraso}
\end{table}



\input{./tables/smallnet_netTE1_stats2.tex}

Para a rede grande, obteve-se o máximo delay de 83.61 micro segundos para o caminho entre Granada e a Corunha, atraso médio de 30.44 micro segundos, e a carga máxima de Aveiro-Porto com 63497 pacotes/ segundo e a carga média de 22735.10 pacotes/segundo.

\begin{table}[!htb]
        \centering
        \resizebox{\textwidth}{!}{\begin{tabular}{|l|l|l|}
\hline
\multicolumn{1}{|c|}{\textbf{Maximum one-way delay flow}} & \multicolumn{1}{|c|}{\textbf{Maximum one-way delay}} & \multicolumn{1}{|c|}{\textbf{Mean one-way delay}} \\ \hline
Granada-Corunha & 83.6110516972 & 30.439503777 \\ \hline
\end{tabular}}
\caption[]{Atraso}
\end{table}



\begin{table}[!htb]
        \centering
        \resizebox{\textwidth}{!}{\begin{tabular}{|l|l|l|}
\hline
\multicolumn{1}{|c|}{\textbf{maxLoadK}} & \multicolumn{1}{|c|}{\textbf{maxLoad}} & \multicolumn{1}{|c|}{\textbf{meanLoad}} \\ \hline
('Aveiro', 'Porto') & 63497 & 22735.1034483 \\ \hline
\end{tabular}}
\caption[]{}
\end{table}



\newpage

\section{Exercício 5, 6 e 7}

No exercício 5 foi pedido que o routing agora fosse feito usando como parâmetro de escolha do caminho mais curto a menor carga possível, otimizando assim a largura de banda disponível  ao longo do caminho.

A nível de código, a única mudança efetuada importante foi:
\begin{lstlisting}[language=python]
path = nx.shortest_path(net, pair[0], pair[1], weight='load')
\end{lstlisting}

Pois agora o critério para escolha do caminho mais curto é a carga.

Para a rede pequena o resultado obtido foi:

\input{./tables/smallnet_netTE2.tex}

Analisando a tabela e os resultados obtidos conseguimos perceber que houve mudança na distribuição dos caminhos mais curtos, por exemplo, Porto > Lisboa agora o caminho é feito por Porto, Viseu e Lisboa, já de Lisboa > Porto é feito por Lisboa, Viseu, Porto.

Este resultado é explicado porque a lista de pares da rede pequena está distribuída da seguinte forma:

\begin{lstlisting}[language=python]
[('Lisboa', 'Viseu'), ('Lisboa', 'Aveiro'),
('Lisboa', 'Porto'), ('Viseu', 'Lisboa'), ('Viseu', 'Aveiro'),
('Viseu', 'Porto'), ('Aveiro', 'Lisboa'), ('Aveiro', 'Viseu'), 
('Aveiro', 'Porto'), ('Porto', 'Lisboa'), ('Porto', 'Viseu'),
('Porto', 'Aveiro')]
\end{lstlisting}

Inicialmente, irá testar a ligação Lisboa > Viseu, à qual será atribuída, depois Lisboa > Aveiro e será também atribuída, de seguida, Lisboa > Porto, como o somatório das cargas entre as ligações Lisboa > Viseu (31271) + Viseu > Porto (0) tem menos carga do que a Lisboa > Aveiro (31345) + Aveiro > Porto (0) então seleciona ir por Viseu, daí a diferença nos resultados em relação ao primeiro exercício.
 
Na diferença de Porto > Lisboa, temos Porto > Aveiro (0)  + Aveiro > Lisboa (31188) com mais carga do que Porto > Viseu (0) + Viseu > Lisboa (31171) no momento de atribuição, o que faz com que seja selecionado o caminho: Porto > Viseu > Lisboa.

\input{./tables/smallnet_netTE2_stats1.tex}

\begin{table}[!htb]
        \centering
        \resizebox{\textwidth}{!}{\begin{tabular}{|l|l|l|}
\hline
\multicolumn{1}{|c|}{\textbf{Max load flow}} & \multicolumn{1}{|c|}{\textbf{Maximum one-way load}} & \multicolumn{1}{|c|}{\textbf{Mean one-way load}} \\ \hline
Viseu-Porto & 62731.00 pkts/sec & 43797.40 pkts/sec \\ \hline
\end{tabular}}
\caption[]{Carga}
\end{table}



Com este exercício, mudando o critério para carga nos links na escolha do caminho mais curto, para a rede pequena, conseguiu-se melhorias no "maximum one-way delay" de -0,102 micro segundos, no atraso médio de -0,0012 micro segundos, no "maximum one-way load" de -319 pkts/sec e a carga média manteve-se como esperado.

Dado que agora o pretendido era querer-se melhorias em termos de carga, estas foram obtidas para a rede pequena.

%Maximum one-way delay delay: 32.1869758326-32.0852532284=0,1017226042 micro sec
%Mean one-way delay: 16.010093405-16.0089473288=0,0011460762 micro sec
%63050.00-62731.00=319 pkts/sec
%43797.40 = 0
\newpage

Para a rede grande, obteve-se o máximo delay de 80.19 micro segundos para o caminho entre Valencia e Badajoz, o que significa uma melhoria de 3,42 micro segundos, já atraso médio de desceu 1,80 micro segundos em relação ao exercício anterior, e obteve-se a carga máxima de Aveiro-Porto com 35315 pacotes/ segundo (melhoria de 28182 pacotes/segundo)  e a carga média de 23310.10 pacotes/segundo (melhoria de 575,1 pacotes/segundo).

%83.6110516972-80.1875239088=3,42
%30.439503777-28.6383060926=1,80
%63497-35315=28182
%22735-23310.10=575,1

\begin{table}[!htb]
        \centering
        \resizebox{\textwidth}{!}{\begin{tabular}{|l|l|l|}
\hline
\multicolumn{1}{|c|}{\textbf{Maximum one-way delay flow}} & \multicolumn{1}{|c|}{\textbf{Maximum one-way delay}} & \multicolumn{1}{|c|}{\textbf{Mean one-way delay}} \\ \hline
Portimao-Porto & 11.1501365892 & 9.93751799553 \\ \hline
\end{tabular}}
\caption[]{Atraso}
\end{table}



\begin{table}[!htb]
        \centering
        \resizebox{\textwidth}{!}{\begin{tabular}{|l|l|l|}
\hline
\multicolumn{1}{|c|}{\textbf{Max load flow}} & \multicolumn{1}{|c|}{\textbf{Maximum one-way load}} & \multicolumn{1}{|c|}{\textbf{Mean one-way load}} \\ \hline
Aveiro-Porto & 35315.00 pkts/sec & 23310.48 pkts/sec \\ \hline
\end{tabular}}
\caption[]{Carga}
\end{table}



\section{Exercício 8, 9 e 10}

No exercício 8 pretende-se que seja escolhido o caminho mais curto tendo em conta o atraso, minimizando assim o "average one-way delay". Este por sua vez irá sempre escolher de forma sequencial, o que fará com que seja sempre na mesma ordem.

A nível de código, a única mudança efetuada importante foi:
\begin{lstlisting}[language=python]
path = nx.shortest_path(net, pair[0], pair[1],
	 weight='delay')
...
net[path[i]][path[i + 1]]['delay'] = 1e6 / 
	(mu - net[path[i]][path[i + 1]]['load'])
\end{lstlisting}

O critério agora é o atraso, e para calcular o atraso usou-se a aproximação M/M/1.

\newpage

Para a rede pequena o resultado obtido foi:

\begin{table}[!htb]
        \centering
        \resizebox{\textwidth}{!}{\begin{tabular}{|l|l|l|l|l|}
\hline
\multicolumn{1}{|c|}{\textbf{Origem}} & \multicolumn{1}{|c|}{\textbf{Destino}} & \multicolumn{1}{|c|}{\textbf{Saltos}} & \multicolumn{1}{|c|}{\textbf{Carga (pkts/sec)}} & \multicolumn{1}{|c|}{\textbf{Atraso (micro/sec)}} \\ \hline
Lisboa & Viseu & Lisboa, Viseu & 62601 & 16.03 \\ \hline
Porto & Lisboa & Porto, Viseu, Lisboa & Indisponível & 15.95 \\ \hline
Viseu & Porto & Viseu, Porto & 62731 & 16.06 \\ \hline
Lisboa & Aveiro & Lisboa, Aveiro & 31345 & 10.68 \\ \hline
Aveiro & Viseu & Aveiro, Viseu & 31199 & 10.66 \\ \hline
Viseu & Aveiro & Viseu, Aveiro & 31378 & 10.68 \\ \hline
Aveiro & Porto & Aveiro, Porto & 31720 & 10.72 \\ \hline
Porto & Viseu & Porto, Viseu & 62512 & 16.00 \\ \hline
Porto & Aveiro & Porto, Aveiro & 31013 & 10.64 \\ \hline
Lisboa & Porto & Lisboa, Viseu, Porto & Indisponível & 16.06 \\ \hline
Viseu & Lisboa & Viseu, Lisboa & 62287 & 15.95 \\ \hline
Aveiro & Lisboa & Aveiro, Lisboa & 31188 & 10.66 \\ \hline
\end{tabular}}
\caption[]{Solução obtida, carga nos links e atraso}
\end{table}



Nesta tabela em relação ao exercício anterior não houve mudanças, isto devido à ordem ser a mesma de atribuição.

\begin{table}[!htb]
        \centering
        \resizebox{\textwidth}{!}{\begin{tabular}{|l|l|l|}
\hline
\multicolumn{1}{|c|}{\textbf{Maximum one-way delay flow}} & \multicolumn{1}{|c|}{\textbf{Maximum one-way delay}} & \multicolumn{1}{|c|}{\textbf{Mean one-way delay}} \\ \hline
Viseu-Porto & 16.0593553775 & 13.3398664587 \\ \hline
\end{tabular}}
\caption[]{Atraso}
\end{table}



\begin{table}[!htb]
        \centering
        \resizebox{\textwidth}{!}{\begin{tabular}{|l|l|l|}
\hline
\multicolumn{1}{|c|}{\textbf{Max load flow}} & \multicolumn{1}{|c|}{\textbf{Maximum one-way load}} & \multicolumn{1}{|c|}{\textbf{Mean one-way load}} \\ \hline
Viseu-Porto & 16.06 pkts/sec & 12.81 pkts/sec \\ \hline
\end{tabular}}
\caption[]{Carga}
\end{table}



Como se pode observar, para a rede pequena manteve-se igual.

Já para a rede grande, como o critério agora é o atraso, espera-se que se tenha obtido melhorias, comparando com o exercício anterior, existiu melhorias, no máximo delay de 0,97 micro segundos para o mesmo caminho do exercício anterior, já o atraso médio desceu também 0,72 micro segundos e obteve-se a carga máxima também de Aveiro ao Porto, piorando em +88 pacotes por segundo sendo que a carga média diminuiu em 557,67 pacotes/segundo.

% Maximum one-way delay: 80.1875239088-79.2209331482=0,9665907606
% Mean one-way delay: 28.6383060926-27.922949817=0,7153562756
% Maximum one-way load: 35315-35403
% Mean one-way load: 23310.48-22752.81=557,67

\begin{table}[!htb]
        \centering
        \resizebox{\textwidth}{!}{\begin{tabular}{|l|l|l|}
\hline
\multicolumn{1}{|c|}{\textbf{Maximum one-way delay flow}} & \multicolumn{1}{|c|}{\textbf{Maximum one-way delay}} & \multicolumn{1}{|c|}{\textbf{Mean one-way delay}} \\ \hline
Portimao-Porto & 11.1610879829 & 9.92986401539 \\ \hline
\end{tabular}}
\caption[]{Atraso}
\end{table}



\begin{table}[!htb]
        \centering
        \resizebox{\textwidth}{!}{\begin{tabular}{|l|l|l|}
\hline
\multicolumn{1}{|c|}{\textbf{Max load flow}} & \multicolumn{1}{|c|}{\textbf{Maximum one-way load}} & \multicolumn{1}{|c|}{\textbf{Mean one-way load}} \\ \hline
Aveiro-Porto & 11.16 pkts/sec & 9.82 pkts/sec \\ \hline
\end{tabular}}
\caption[]{Carga}
\end{table}



\section{Exercício 11 e 12}

No exercício 11 e 12, foi pedido que usando um método que aleatoriamente trocasse os elementos de uma lista e fazendo isto para um número de vezes finito, de forma a obter sempre os melhores resultados tendo em conta o atraso. Para isto, foi usado o código do exercício anterior, adicionado um for para iterar e mais umas pequenas mudanças.

\begin{lstlisting}[language=python]
# antes de iterar e preciso declarar os dicionarios, 
# listas e rede melhores
allpairs_best = []
sol_best = {}
ws_delay_best = {}
liststats_result = None
net_best = nx.DiGraph()

# iterar 10000x
for i in range(0, 10000):
	... # igual ao exercicio anterior
	
	# ver se e a melhor solucao
	tmp_stats = listStats(ws_delay)

    # best solution
    if liststats_result is None or 
    		liststats_result[0] > tmp_stats[0]:
        allpairs_best = allpairs
        sol_best = sol.copy()
        ws_delay_best = ws_delay.copy()
        liststats_result = tmp_stats
        net_best = net_tmp.copy()
\end{lstlisting}

\newpage
Para a rede pequena o resultado obtido foi:

\begin{table}[!htb]
        \centering
        \resizebox{\textwidth}{!}{\begin{tabular}{|l|l|l|l|l|}
\hline
\multicolumn{1}{|c|}{\textbf{Origem}} & \multicolumn{1}{|c|}{\textbf{Destino}} & \multicolumn{1}{|c|}{\textbf{Saltos}} & \multicolumn{1}{|c|}{\textbf{Carga (pkts/sec)}} & \multicolumn{1}{|c|}{\textbf{Atraso (micro/sec)}} \\ \hline
Lisboa & Viseu & Lisboa, Viseu & 62470 & 15.99 \\ \hline
Porto & Lisboa & Porto, Viseu, Lisboa & Indisponível & 31.95 \\ \hline
Viseu & Porto & Viseu, Porto & 63121 & 16.16 \\ \hline
Lisboa & Aveiro & Lisboa, Aveiro & 62675 & 16.04 \\ \hline
Aveiro & Viseu & Aveiro, Lisboa, Viseu & 31720 & 31.96 \\ \hline
Viseu & Aveiro & Viseu, Aveiro & 31378 & 10.68 \\ \hline
Aveiro & Porto & Aveiro, Viseu, Porto & 31330 & 26.88 \\ \hline
Porto & Viseu & Porto, Viseu & 62512 & 16.00 \\ \hline
Porto & Aveiro & Porto, Aveiro & 31013 & 10.64 \\ \hline
Lisboa & Porto & Lisboa, Aveiro, Porto & Indisponível & 26.72 \\ \hline
Viseu & Lisboa & Viseu, Lisboa & 62287 & 15.95 \\ \hline
Aveiro & Lisboa & Aveiro, Lisboa & 62387 & 15.97 \\ \hline
\end{tabular}}
\caption[]{Solução obtida, carga nos links e atraso}
\end{table}



Em relação aos exercícios anteriores, esta solução é diferente. Dado que agora a lista é re-ordenada de forma aleatória 10 mil vezes, possivelmente terão sido testadas todas as possibilidades para a rede pequena.

\input{./tables/smallnet_netTE4_stats1.tex}

\begin{table}[!htb]
        \centering
        \resizebox{\textwidth}{!}{\begin{tabular}{|l|l|l|}
\hline
\multicolumn{1}{|c|}{\textbf{Maximum one-way delay}} & \multicolumn{1}{|c|}{\textbf{Mean one-way delay}} & \multicolumn{1}{|c|}{\textbf{Max delay flow}} \\ \hline
16.16 pkts/sec & 13.35 pkts/sec & Aveiro-Porto \\ \hline
\end{tabular}}
\caption[]{Carga}
\end{table}



Analisando os resultados obtidos, em relação ao exercício anterior, conseguiu-se uma melhoria no atraso médio de -0,012 micro segundos. Para os restantes parâmetros, os resultados foram iguais.

% maximum one-way delay: 32.0852532284-32.0852532284=0 micro segundos
% mean one-way delay: 16.0089473288-15.9968868727=0,012 micro segundos
% maximum one-way load: 62731-62731=0 pacotes/segundo
% mean one-way load: 43797.40-43797.40=0 pacotes/segundo

\begin{table}[!htb]
        \centering
        \resizebox{\textwidth}{!}{\begin{tabular}{|l|l|l|}
\hline
\multicolumn{1}{|c|}{\textbf{Maximum one-way delay flow}} & \multicolumn{1}{|c|}{\textbf{Maximum one-way delay}} & \multicolumn{1}{|c|}{\textbf{Mean one-way delay}} \\ \hline
Palma-Vigo & 59.1836392733 & 27.1095094606 \\ \hline
\end{tabular}}
\caption[]{Atraso}
\end{table}



\begin{table}[!htb]
        \centering
        \resizebox{\textwidth}{!}{\begin{tabular}{|l|l|l|}
\hline
\multicolumn{1}{|c|}{\textbf{Max load flow}} & \multicolumn{1}{|c|}{\textbf{Maximum one-way load}} & \multicolumn{1}{|c|}{\textbf{Mean one-way load}} \\ \hline
Vilar.Formoso-Braganca & 30905.00 pkts/sec & 22172.22 pkts/sec \\ \hline
\end{tabular}}
\caption[]{Carga}
\end{table}



% maximum one-way delay: 79.2209331482-60.0122049529=19,21 micro segundos
% mean one-way delay: 27.922949817-26.934401214=0,99 micro segundos
% maximum one-way load:  35403.00-30905.00=4498 pacotes/segundo
% mean one-way load: 22752.81-22172.22=580,59 pacotes/segundo

As melhorias também apareceram na rede grande, como agora se fez muitas repetições e pretende-se melhorar o atraso médio, comparando com o exercício anterior, existiu melhorias, no máximo delay de 19,21 micro segundos para o mesmo caminho do exercício anterior, já o atraso médio desceu também 0,99 micro segundos e obteve-se a carga máxima também de Aveiro ao Porto, melhorando em -4498 pacotes por segundo, sendo que a carga média diminuiu também em 580,59 pacotes/segundo.

\section{Exercício 13}

Neste exercício, é pedido que se implemente um algoritmo de procura local de forma a obter a melhor solução minimizando o atraso médio da rede.

Pretende-se que usando o código desenvolvido anteriormente se encontre uma solução inicial que depois será usada pelo algoritmo de procura local para encontrar uma melhor solução, sempre que essa solução encontrada for melhor do que a solução melhor atual esta será substituída pela nova solução.

Portanto o código inicial é igual ao do exercício 8, depois é guardada esta solução que será usada como referência pelo algoritmo de procura local. 

\begin{lstlisting}[language=python]
...

# antes de iterar e preciso declarar os dicionarios, 
# listas e rede melhores encontradas
allpairs_best = []
sol_best = {}
ws_delay_best = {}
liststats_result = None
net_best = nx.DiGraph()

# iterar pela lista de pares
for pair in allpairs:
	# copiar a solucao e a net encontrada anteriormente
	sol_tmp = sol.copy()
    net_tmp = net.copy()
    
    # coloca-se o dicionario do atraso vazio
    ws_delay = {}
	
	# caminho para o par selecionado
    path = sol_tmp[pair]
    # apagar da solucao o par selecionado
    del sol_tmp[pair]
    
    # para o caminho para o par selecionado
    for i in range(0, len(path) - 1):
    	# remover a carga
        net_tmp[path[i]][path[i + 1]]['load'] -= tm[pair[0]][pair[1]]
        # recalcular o atraso
        net_tmp[path[i]][path[i + 1]]['delay'] = 1e6 / (mu - \
        	net_tmp[path[i]][path[i + 1]]['load'])
		
	# faz-se a procura do caminho mais pequeno de novo
    	path = nx.shortest_path(net_tmp, pair[0], pair[1],
    					 weight='delay')
    	# grava-se esse caminho na nova solucao
    	sol_tmp.update({pair: path})		
    	
    	....
    	# repoe-se a carga para o caminho novo encontrado
    	# recalcula-se o atraso
    	
    	# calula-se os atrasos (ws_delay)
    	
    	# verifica-se se e a melhor solucao e se sim, substitui-se
\end{lstlisting}

Para a rede pequena o resultado obtido foi:

\begin{table}[!htb]
        \centering
        \resizebox{\textwidth}{!}{\begin{tabular}{|l|l|l|l|l|}
\hline
\multicolumn{1}{|c|}{\textbf{Origem}} & \multicolumn{1}{|c|}{\textbf{Destino}} & \multicolumn{1}{|c|}{\textbf{Caminho}} & \multicolumn{1}{|c|}{\textbf{Carga (pkts/sec)}} & \multicolumn{1}{|c|}{\textbf{Atraso (micro/sec)}} \\ \hline
Lisboa & Viseu & Lisboa, Viseu & 62601 & 16.03 \\ \hline
Porto & Lisboa & Porto, Aveiro, Lisboa & Indisponível & 31.86 \\ \hline
Viseu & Porto & Viseu, Porto & 62731 & 16.06 \\ \hline
Lisboa & Aveiro & Lisboa, Aveiro & 31345 & 10.68 \\ \hline
Aveiro & Viseu & Aveiro, Viseu & 31199 & 10.66 \\ \hline
Viseu & Aveiro & Viseu, Aveiro & 31378 & 10.68 \\ \hline
Aveiro & Porto & Aveiro, Porto & 31720 & 10.72 \\ \hline
Porto & Viseu & Porto, Viseu & 31396 & 10.68 \\ \hline
Porto & Aveiro & Porto, Aveiro & 62129 & 15.91 \\ \hline
Lisboa & Porto & Lisboa, Viseu, Porto & Indisponível & 32.09 \\ \hline
Viseu & Lisboa & Viseu, Lisboa & 31171 & 10.66 \\ \hline
Aveiro & Lisboa & Aveiro, Lisboa & 62304 & 15.95 \\ \hline
\end{tabular}}
\caption[]{Solução obtida, carga nos links e atraso}
\end{table}



\begin{table}[!htb]
        \centering
        \resizebox{\textwidth}{!}{\begin{tabular}{|l|l|l|}
\hline
\multicolumn{1}{|c|}{\textbf{Maximum one-way delay flow}} & \multicolumn{1}{|c|}{\textbf{Maximum one-way delay}} & \multicolumn{1}{|c|}{\textbf{Mean one-way delay}} \\ \hline
Lisboa-Porto & 32.0852532284 & 15.9968868727 \\ \hline
\end{tabular}}
\caption[]{Atraso}
\end{table}



\input{./tables/smallnet_netTE5_stats2.tex}

Analisando os resultados obtidos, em relação ao exercício anterior, conseguiu-se uma melhoria no atraso médio de -0,012 micro segundos. Para os restantes parâmetros, os resultados foram iguais.

% maximum one-way delay: 32.0852532284-32.0852532284=0 micro segundos
% mean one-way delay: 16.0089473288-15.9968868727=0,012 micro segundos
% maximum one-way load: 62731-62731=0 pacotes/segundo
% mean one-way load: 43797.40-43797.40=0 pacotes/segundo

Já para a rede grande, como o critério agora é o atraso, espera-se que se tenha obtido melhorias, comparando com o exercício anterior, existiu melhorias, no máximo delay de 0,97 micro segundos para o mesmo caminho do exercício anterior, já o atraso médio desceu também 0,72 micro segundos e obteve-se a carga máxima também de Aveiro ao Porto, piorando em +88 pacotes por segundo sendo que a carga média diminuiu em 557,67 pacotes/segundo.

% Maximum one-way delay: 80.1875239088-79.2209331482=0,9665907606
% Mean one-way delay: 28.6383060926-27.922949817=0,7153562756
% Maximum one-way load: 35315-35403
% Mean one-way load: 23310.48-22752.81=557,67

\begin{table}[!htb]
        \centering
        \resizebox{\textwidth}{!}{\begin{tabular}{|l|l|l|}
\hline
\multicolumn{1}{|c|}{\textbf{Maximum one-way delay flow}} & \multicolumn{1}{|c|}{\textbf{Maximum one-way delay}} & \multicolumn{1}{|c|}{\textbf{Mean one-way delay}} \\ \hline
Palma-Vigo & 60.3428511256 & 27.6466866517 \\ \hline
\end{tabular}}
\caption[]{Atraso}
\end{table}



\input{./tables/network_netTE5_stats2.tex}



\section{Exercício 14}



\end{document}



